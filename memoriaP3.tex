\documentclass{article}
\usepackage[left=3cm,right=3cm,top=2cm,bottom=2cm]{geometry} % page
                                                             % settings
\usepackage{amsmath} % provides many mathematical environments & tools
\usepackage[spanish]{babel}
\usepackage[doument]{ragged2e}

% Images
\usepackage{graphicx}
\usepackage{float}

% Code
\usepackage{listings}
\usepackage{xcolor}
\definecolor{gray}{rgb}{0.5,0.5,0.5}
\newcommand{\n}[1]{{\color{gray}#1}}
\lstset{numbers=left,numberstyle=\small\color{gray}}

\selectlanguage{spanish}
\usepackage[utf8]{inputenc}
\setlength{\parindent}{0mm}



\usepackage{listings}
\lstset
{ %Formatting for code in appendix
  language=C++, % choose the language of the code
  basicstyle=\fontfamily{pcr}\selectfont\footnotesize\color{black},
  keywordstyle=\color{darkorange}\bfseries, % style for keywords
  numbers=left, % where to put the line-numbers
  numberstyle=\tiny, % the size of the fonts that are used for the line-numbers     
  backgroundcolor=\color{white},
  showspaces=false, % show spaces adding particular underscores
  showstringspaces=false, % underline spaces within strings
  showtabs=false, % show tabs within strings adding particular underscores
  tabsize=2, % sets default tabsize to 2 spaces
  captionpos=b, % sets the caption-position to bottom
  breaklines=false, % sets automatic line breaking
  breakatwhitespace=false, 
}

\definecolor{darkorange}{rgb}{0.94,0.4,0.0}

\begin{document}

\title{Poda alfa-beta con heurística}
\author{David Cabezas Berrido}
\date{\today}
\maketitle
\tableofcontents

\newpage

\section{Objetivo}

El objetivo es programar una inteligencia artificial que juegue a Desconecta 4 Boom, y que use el algoritmo de poda alfa-beta con una heurística. \\

En la memoria explicaré como he implementado el algoritmo y la
heurística, así como alguna modificación que he realizado.


\section{Algoritmo de Poda alfa-beta}

Es una variante del Minimax un algoritmo muy eficaz para explorar los
caminos que puede seguir una partida en un juego de dos jugadores como
ajedrez, damas o Desconecta 4 Boom. Se basas en explorar un árbol de
posibilidades en el que en los niveles pares juega un jugador y es los
impares otro, se supone que cada jugador realiza el mejor movimiento
posible. Por tanto, si tenemos una valoración de un estado del juego
que indica la ventaja de uno de los dos jugadores, ese jugador hará el
movimiento que maximice esa valoración y su oponente el movimiento que
la minimice. \\

El algoritmo de Poda alfa-beta incluye una modificación para evitar
explorar ramas del árbol que presentan una desventaja evitable para
uno de los jugadores, luego el juego jamás se desarrollará por ese
camino. Para ello se almacenan:
\begin{itemize}
\item El la variable alpha, la puntuación mínima que tiene asegurada
  el jugador que maximiza. Se inicializa a $-\infty$.
\item En la variable beta, la puntuación máxima que tiene asegurada el
  jugador que minimiza. Se inicializa a $+\infty$.
\end{itemize}

Por tanto, en el momento que alpha supere a beta se estará explorando
un camino en el que un jugador tiene peor puntuación (para él) de la
que tenía asegurada y evitará que el juego siga esa rama. \\

Nos es imposible explorar el árbol infinitamente, así que exploramos
hasta profundidad 8 (4 jugadas de cada jugador por delante de al
situación actual) o hasta que acabe el juego. Entonces devolvemos la
valoración heurística del tablero. Un jugador intentará llegar al nodo
con mayor valoración heurística y el otro al nodo con menor valoración
heurística. \\

He implementado este algoritmo en la función \texttt{Poda\_AlphaBeta},
que devuelve al jugador que la llama la puntuación óptima que puede
conseguir (asegurada). El metodo \texttt{Think} toma la lista de
tableros a los que se pueden llegar en la siguiente acción desde el
tablero actual y evalúa la función \texttt{Poda\_AlphaBeta} para cada
uno de esos tableros, eligiendo la que ofrezca mejor puntuación. \\

En la implementación que he hecho, siempre considero que el jugador al
que le toca es el que maximiza, por lo que cuando se llama a la
\texttt{Valoracion}, siempre se llama con el jugador que llamó a la
función en primera instancia (en el método \texttt{think}).


\section{Función Valoración}

La función \texttt{Valoracion} asigna un valor heurístico a un tablero
para un jugador, y devuelve un valor que es mayor cuanto mayor sea la
ventaja de ese jugador en ese tablero. Lo que tiene en cuenta es:

\begin{itemize}
\item \textbf{Si la partida ha terminado:} devuelve $-\infty$ si el
  jugador que solicita la valoración ha perdido, $+\infty$ si ha
  ganado y 0 si hay empate.

\item \textbf{El número de 3 en raya:} cuenta el número de 3 en raya
  que tiene cada jugador. Usa la función \texttt{NenLinea3}, que se
  auxilia de \texttt{EnLinea3} una modificación del método
  \texttt{Enlinea} de la clase \texttt{Environment}, que comprueba si
  hay un 4 en raya. A la diferencia del número de 3 en raya le asigna
  un peso \texttt{PESO3L}, esto resultará en un número positivo si
  el enemigo tiene más 3 en raya que nosotros y en un número negativo
  en caso contrario.

\item \textbf{El número de 2 en raya:} cuenta el número de 2 en raya
  que tiene cada jugador usando las funciones \texttt{NenLinea2} y
  \texttt{EnLinea2} de forma análoga al número de 3 en raya. Y le
  asigna un peso \texttt{PESO2L} a la diferencia. Hay que tener
  presente que un 3 en raya va a contar como dos 2 en raya.

\item \textbf{El número de fichas} de cada jugador. A la diferencia le
  asigna un peso \texttt{PESOFICHA}.

\item \textbf{La buena colocación de la bomba:} este valor ya es mayor
  cuanto mejor colocada esté la bomba del jugador que pide la
  valoración respecto a la del enemigo y le da un peso de
  \texttt{PESOBOMBA}. Si no hay bomba devuelve 0, en otro caso:
  \begin{itemize}
  \item La puntuación es mayor cuanto más baja esté la bomba, le da
    una importancia \texttt{PESOBAJO}.
  \item Es mayor cuanto más fichas del propio jugador van a ser
    destruidas por la bomba, multiplica por \texttt{PESOELIMINAR}.
  \item Es mayor cuando más fichas tenga el adversario en la fila
    superior a la bomba, habrá mayor probabilidad de que sus fichas
    caigan cuando la bomba explote, multiplica por
    \texttt{PESODESPLAZAR}.
  \end{itemize}
\end{itemize}

Ya sólo queda jugar con los pesos (en \texttt{jugador.h}) e ir
analizando el comportamiento del jugador.


\section{Problema al equilibrar el peso de la bomba}

Le he dado mucha importancia a colocar la bomba bien, ya que en la
práctica daba buenos resultados. El problema es que la puntuación que
ofrecía a un tablero el tener la bomba bien colocada, era mayor que la
puntuación que recibía por la ventaja que producía explotarla. Esto
resultaba en que nunca explotaba la bomba. \\

En cambio, el hecho de dar poco peso a la colocación de la bomba
producía en que se la tratara como una ficha normal, y se evitase
colocar alineada con el resto de fichas. Esto provoca que al explotar
la bomba no se gane mucha ventaja. \\

Esto podría solucionarse haciendo más pruebas hasta encontrar el
equilibrio en los pesos deseado, pero hacer una prueba consume mucho
tiempo. También podría solucionarse si tuviésemos un horizonte de
búsqueda mayor, para que contase con la bomba próxima o visualizase la victoria. \\

Como ninguna de estas soluciones era factible, ya fuera por mi tiempo
limitado o por las limitaciones de la práctica, he añadido una
excepción en la regla del algoritmo de Poda alfa-beta. \\

Si después de calcular las valoraciones (utilizando la Poda alfa-beta)
de las distintas acciones se dan una serie de características, se
elige la acción BOOM, ya que la puntuación que recibe por la función
valoración no hace justicia a lo adecuada que es. Las condiciones son:
\begin{itemize}
\item La acción BOOM está disponible (tengo una bomba colocada).
\item El número de jugada es congruente con 4 módulo 5 (si no detono
  la bomba ahora, perderé la siguiente).
\item La acción mejor valorada tiene una valoración por debajo de
  cierto umbral (puede haber una alternativa muy buena).
\item La acción BOOM tiene una valoración por encima de cierto umbral
  (la acción BOOM no me lleva a una derrota casi segura).
\end{itemize}

Esto rompe con el espíritu del algoritmo de Poda alfa-beta, pero es
una solución muy simple y rápida al problema y da muy buenos
resultados. Con este algoritmo he conseguido vencer a los tres ninjas
tanto como primer jugador, como con el segundo.

\end{document}
